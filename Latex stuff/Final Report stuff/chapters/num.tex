In this chapter, we discuss some interesting numerical methods to
solve the Hamilton Jacobi Equations. First we start off with a brief
introduction to Hyperbolic Conservation Laws. In the next section, we
discuss some of the numerical techniques to solve Hyperbolic
Conservation Laws. In section 3, we discuss the numerical schemes to
solve Eikonal Type HJE, with the idea of conservation laws. Next, we discuss the numerical schemes to solve the Perspective Projection Model developed by Prados and Faugeras\cite{prados2}.


\section{Hyperbolic Conservation Laws}
Scalar Hyperbolic Conservation Laws in 1D is represented as
\begin{equation}
	u_t + f(u)_x = 0 \label{eq:24}
\end{equation}
where $f:\mathbb{R} \to \mathbb{R}$ is a smooth function known as Flux. If $u$ is a smooth function ,(\ref{eq:24}) can be written in a non-conservative form,
\begin{equation}
	u_t + f'(u)u_x = 0\label{eq:25}
\end{equation}
(\ref{eq:25}) when augmented with initial conditions $u(x,0) = u_0(x)$, can be solved with the \textbf{method of characteristics}. This is possible only if $f$ is linear. If $f$ is nonlinear, we have non-smooth solutions even if $f$ and $u_0$ are smooth. On using method of characteristics, we observe that the characteristics intersect at some point say P. At P the solution $u$ takes multiple values. These points where uniqueness breakdown are known as ``shocks". Thus it becomes insufficient to define the solution with the classical theory. So we are in a point to define ``weak solutions" to the PDE. For more details, we refer the reader to \cite{leve}.
\begin{definition}
	\textbf{Weak solution}\\
	
	\noindent
	Let $v : \mathbb{R} \times [0,\infty) \to \mathbb{R} $, be a smooth function with compact support. A bounded measurable function $u$ is said to be a weak solution to (\ref{eq:24}) if
	\begin{equation}
		\int_{0}^{\infty} \int_{-\infty}^{\infty} \left(uv_t + f(u)v_x\right) dxdt + \int_{-\infty}^{\infty} u v|_{t=0}dx = 0 \label{eq:26}
	\end{equation}
\end{definition}

\noindent
This is obtained directly by multiplying (\ref{eq:24}) with $v$ and then integrating over $R \times [0,\infty)$. If $u$ is a smooth and satisfies (\ref{eq:26}), then $u$ satisfies (\ref{eq:24}). We can show that the shock propogates with a speed $s$ along the discontinuity where,
\begin{equation}
	s = \frac{[f(u)]}{[u]}
\end{equation} 
and $[f(u)] = f(u_r) - f(u_l)$ is the jump in $f(u)$ across the discontinuity, $[u] = u_r - u_l$ is the jump in $u$ across the discontinuity. This condition is known as the \emph{Rankine-Hugniot Condition}. This states that, no matter what values $u_l$ and $u_r$ might take, the RH Condition must always be satisfied at the shock.\\

\noindent
But for (\ref{eq:24}), we may have more than one solution satisfying the RH-Condition. But only one solution is the physically relevant solution. This is known as the Entropy Solution (Analogous to the viscosity solution defined in Chapter 2). An entropy solution satisfies what is known as the entropy condition, which is defined below.
\begin{definition}
	\textbf{Entropy condition for convex flux}\\
	
	\noindent
	A discontinuity propogating with a speed $s$ satisfying the RH condition, satisfies the entropy condition if 
	\begin{equation}
		f'(u_l) > s > f'(u_r)
	\end{equation}
	which is equivalent to $u_l > u_r$ if $f$ is convex.
\end{definition}
\begin{definition}
	\textbf{Kruzkov Entropy condition}\\
	
	\noindent
	A discontinuity propogating with a speed $s$ satisfying the RH condition, satisfies the entropy condition if 
	\begin{equation}
	\frac{f(v) - f(u_l)}{v-u_l} > s > \frac{f(v) - f(u_r)}{v - u_r}
	\end{equation}
	This will hold for any flux function $f$.
\end{definition}

\noindent
Next, we define the \emph{Riemann Problem}. This will be used to construct the numerical schemes, where local Riemann Problems are solved in a finite volume and then patched up to get the solution to our problem.\\

\noindent
A conservation law, together with a piecewise constant data having a single discontinuity is known as a Riemann Problem.
\begin{eqnarray}
	u_t + f(u)_x &=& 0 \\
	u(x,0) &=& \left\{ 
		\begin{array}{ll}
			u_l  & x < 0 \\
			u_r & x > 0
		\end{array} 
		\right.
\end{eqnarray}

\noindent
The solution to the Riemman problem for any flux $f$ is given by\cite{leve},
\begin{eqnarray}
	u(x,t) &=& \left\{ 
	\begin{array}{ll}
	u_l  & x < tf'(u_l) \\
	(f')^{-1}\left(\frac{x}{t}\right) & tf'(u_l) < s < tf'(u_r)\\
	u_r & x > tf'(u_r)
	\end{array} 
	\right.
\end{eqnarray}

\section{Numerical Schemes for Hyperbolic Conservation Laws}
In this section, we see how to develop Finite Difference approximations to Hyperbolic conservation laws. First let us consider the problem,
\begin{eqnarray}
	u_t + f(u)_x &=& 0 \qquad \text{in} \;\; \mathbb{R} \times (0,\infty)\label{eq:28}\\
	u(x,0) &=& u_0(x) \qquad x \in \mathbb{R}\label{eq:29}
\end{eqnarray}

\noindent
Now we discretize the $x$ axis by $\left\{x_{i+\frac{1}{2}}\right\}$, where
\begin{equation}
	x_{i+\frac{1}{2}} = \left(i + \frac{1}{2}\right) h, \quad i \in \mathbb{Z}, h > 0
\end{equation}
and $t$ by $\left\{t_n\right\}$, where
\begin{equation}
	t_n = n \Delta t \quad n = 0,1,2,\dots
\end{equation}
Let $\lambda = \Delta t / h$. Let $v_i^n$ denotes the approximate solution at the point $\left(x_{i+\frac{1}{2}}, t_n\right)$. Thus the initial data $\{v_i^0\}$ is given by
\begin{equation}
	v_i^0 = \frac{1}{h} \int_{x_{i-1/2}}^{x_{i+1/2}} u_0(x) dx
\end{equation}

\noindent
Now, we define the following terms,
\begin{definition}
	A finite difference scheme is said to be in the conservative form, if there exists a continuous function $F: \mathbb{R}^{2k} \to \mathbb{R}$ such that
	\begin{equation}
		v_i^{n+1} = H(v_{i-k}^n, \dots,v_{i+k}^n) = v_i^n - \lambda \left(F_{i+1/2}^n - F_{i-1/2}^n\right)\label{eq:27}
	\end{equation}
	where $F_{i+1/2} = F(v_{i-k+1}^n, \dots, v_{i+k}^n )$. The function $F$ is called the numerical flux.
\end{definition}
\begin{definition}
	A finite difference scheme is said to be consistent with the conservation law if 
	\begin{equation}
		F(v,\dots,v) = f(v) \quad \forall v \in \mathbb{R}
	\end{equation}
\end{definition}

\noindent
For the convergence of a scheme to the entropy solution, Lax-Wendroff theorem\cite{gowda} says that the scheme must be in the conservative form and consistent. This ensures that the approximate solution $v_i^n$ converges to the entropy solution. For the proof of the theorem, we refer the reader to \cite{gowda}.

\subsection{The Godunov Scheme}
The Godunov Scheme for the numerical solution conservation law defined in (\ref{eq:28}) - (\ref{eq:29}) is given by \cite{gowda,leve},
\begin{eqnarray}
	v_j^{n+1} &=& v_j^{n} - \lambda \left(F^n_{j+1/2} - F^n_{j-1/2}\right)\label{conlaw}\\
	\text{where} \qquad F^n_{j+1/2} &=& f(w_R(0,v_j^n,v_j^{n+1}))
\end{eqnarray}
where $w_R$ is the solution to the Riemann Problem.\\

\noindent
For a convex flux $f$, the Godunov Flux is given by,
\begin{equation}
	F^G(u,v) = \max(f(\max(u,\theta)), f(\min(v,\theta))) \label{godunov}
\end{equation}
and for a concave flux $f$,
\begin{equation}
	F^G(u,v) = \min(f(\min(u,\theta)), f(\max(v,\theta)))
\end{equation}

\noindent
where $\theta = \min_{u \in \mathbb{R}} f(u)$. This scheme is stable under the CFL - Condition\cite{leve}
\begin{equation}
	\lambda \sup_{j} v_j \le 1 \label{cfl}
\end{equation}

\noindent
For the detailed derivation of the Godunov Scheme, one can refer \cite{leve}. We can show that the Godunov scheme is in the conservative form, is consistent, stable under the CFL condition and hence converges to the entropy solution and is a upwind scheme.\cite{leve}

\subsection{Lax-Friedrich Scheme}
For the conservative form of the scheme in (\ref{conlaw}), the \textbf{Lax Friedrich} Flux is given by
\begin{equation}
	F^{LF} (u,v) = \frac{1}{2} \left(f(u) + f(v) - \frac{1}{\lambda} (v - u)\right) \label{laxf}
\end{equation}

\noindent 
The Lax-Friedrich Scheme is consistent and stable under the CFL Condition (\ref{cfl}). Thus this scheme converges to the entropy solution. But Lax-Friedrich Scheme is not an upwind scheme.

\section{Godunov Scheme for the Eikonal Equation}
In this section, we see how the Eikonal Equation in 1D can be viewed as a conservation law, thus enabling us to use the schemes defined in the previous section.\\

\noindent
The Eikonal Equation in 1D is given by, 
\begin{eqnarray}
	\lvert u_x \rvert = 1
\end{eqnarray}

\noindent
To solve this equation, we add a transient term $u_t$ and solve it till the steady state is reached.
\begin{equation}
	u_t + \lvert u_x \rvert = 1 \label{eiko}
\end{equation}

\noindent
where $u = u(x,t)$. Now set $v = u_x$. Differentiating (\ref{eiko}) paritally with respect to $x$, we get
\begin{eqnarray}
	&(u_t)_x& + \;\;(\lvert u_x \rvert)_x = 0\nonumber\\
	\implies &(u_x)_t& + \;\;(\vert u_x \rvert)_x = 0\nonumber\\
	\implies & v_t& + \;\;(\lvert v \rvert)_x = 0 \label{conlaw2}
\end{eqnarray}

\noindent
(\ref{conlaw2}) is precisely the conservation law, with $f(u) = \lvert u \rvert$. On using the Godunov scheme defined by (\ref{conlaw}), we have
\begin{equation}
	u_i^{n+1} = u_i^{n}	 - \Delta t \;\left(F^G\left(0,\frac{u_{i}^n - u_{i-1}^n}{h},\frac{u_{i+1}^n - u_{i}^n}{h}\right) -1 \right)
\end{equation}

\noindent
As $f(u) = \lvert u \rvert$ is a convex function, we can say that
\begin{equation}
	F^G(u,v) = \max(f(\max(u,0)), f(\min(v,0)))
\end{equation}

\noindent
This idea can be extended to the 2D case as well. For the Eikonal Equation
\begin{eqnarray}
	u_t + \sqrt{\left(\frac{\partial u }{\partial x}\right)^2 + \left(\frac{\partial u }{\partial y}\right)^2}  = 1
\end{eqnarray}
we can use the Godunov Flux to approximate the derivatives along each direction. So the scheme would become
\begin{eqnarray}
	u_{i,j}^{n+1} = u_{i,j}^n - \Delta t \; \left(\sqrt{F^{G_x}(0,D^-x,D^+x)^2 + F^{G_y}(0,D^-y,D^+y)^2} - 1\right)
\end{eqnarray}
where $D^\pm x$ and $D^\pm y$ denotes the forward and backward differences along $x$ and $y$ respectively. This scheme is consistent, stable under the CFL condition $\Delta t \le \min(h_x,h_y)$. \\

\noindent
This can be extended to the Hamiltonian defined by Prados in \cite{prados2} as well. The derivatives have to be replaced by the appropriate Fluxes to get the explicit scheme. The second term can be upwinded ($D^x$ and $D^y$) according to the derivative formula chosen by the Godunov Flux. One can easily verify that the scheme to solve the model is given by,
	\begin{equation}
		v_{i,j}^{n+1} = v_{i,j}^n - \Delta t \left(-e^{-2v_{i,j}} + J(x_i,y_j) \sqrt{f^2\left( \left(F^{G_x}\right)^2 + \left(F^{G_y}\right)^2\right)  + \left(x_iD^x + y_jD^y\right)^2 + Q(x_i,y_j)^2} \right) \label{eq:upw}
	\end{equation}\\

\noindent
The upwinding can be taken one-step further by taking $J(x,y)$ inside the square-root and upwinding $J(x,y)$ along with the derivatives. The results seem to be better when upwinding is done with $J(x,y)$ included inside the square root (Shown in Chapter 6) compared with the standard upwinding. In view of brevity, we just illustrate how the Godunov flux is modified.
\begin{equation}
	J_{i,j} \;F^{G_x}(0,D^+x,D^-x) \implies  F^{G_x}(0,J_{i+1,j}D^+x,J_{i,j}D^-x) \label{eq:upw1}
\end{equation}
where $J_{i,j} = J(x(i),y(j))$. Similarly upwinding is done in the $y$ direction. The second and the third term is multiplied by $J_{i,j}$ alone. 